\documentclass[11pt]{amsart}
\usepackage{geometry}                % See geometry.pdf to learn the layout options. There are lots.
\geometry{a4paper}                   % ... or a4paper or a5paper or ... 
%\geometry{landscape}                % Activate for for rotated page geometry
%\usepackage[parfill]{parskip}    % Activate to begin paragraphs with an empty line rather than an indent
\usepackage{graphicx}
\usepackage{amssymb}
\usepackage{epstopdf}
\DeclareGraphicsRule{.tif}{png}{.png}{`convert #1 `dirname #1`/`basename #1 .tif`.png}

\usepackage{listings}
\usepackage{color}

\definecolor{dkgreen}{rgb}{0,0.6,0}
\definecolor{gray}{rgb}{0.5,0.5,0.5}
\definecolor{mauve}{rgb}{0.58,0,0.82}
%frame=tb
\lstset{frame=none ,
  language=Java,
  aboveskip=3mm,
  belowskip=3mm,
  showstringspaces=false,
  columns=flexible,
  basicstyle={\small\ttfamily},
  numbers=none,
  numberstyle=\tiny\color{gray},
  stringstyle=\color{mauve},
  keywordstyle=\color{blue},
  commentstyle=\color{dkgreen},
  stringstyle=\color{mauve},
  breaklines=true,
  breakatwhitespace=true,
  tabsize=3
}

\title{Design Patterns}
\author{Object Oriented Modeling and Design}
%\date{}                                           % Activate to display a given date or no date


%--------------------------------------------------------------------------------------------------------
%DOCUMENT START
%--------------------------------------------------------------------------------------------------------
\begin{document}
\maketitle
\lstset{language=Java}

\section{List of Patterns}
\begin{itemize}
\item Visitor
\item Command
\item Composite
\item Template/mallmetod (slides lec. 4)
\item Strategy/strategi (slides lec. 4)
\item State
\item Null Object % needed for exercise 3
\item Decorator % needed for exercise 3
\item MVC (Model View Control-archetecture) % needed for exercise 3 (observer
% framework)
\item Observer (slides lec. 6)
\end{itemize}

\section{Visitor}

\section{Command}

\section{Composite}

\section{Template}

\section{Strategy}
\begin{itemize}
  \item Compared with template:
  	\begin{itemize}
  	  \item Both patterns are used to eleminate duplacate code
  	  \item Use \underline{template when the functionality will be the same} for
  	  the whole life span of the object.
  	  \item Use \underline{strategy when the functionality needs to change} for.
  	\end{itemize}
  	\item Strategy compared with state: (slide lec.4 good diagrams)
	\item Stratigies don't have states.
\end{itemize}

\begin{lstlisting}
//two strategy interfaces

public interface Tariff {
	public Money price(int zones);
}

public interface Rebate {
	public Money price(double price);
}

//two strategy implementations

public class Skane implements Tariff { 
	public Money price(int zones) {
		switch (zones) { 
			case 1: return 17.0; 
			default: return 7.0 * zones + 7.0;
		} 
	}
}

public class Family implements Rebate {
	public Money price(double price) { 
		return 1.5 *price;
	} 
}
\end{lstlisting}

\section{State}

\section{Null Object}
% lec. 6

\section{Decorator}
% lec. 5 & 6 slides
% difference between decorator and composite lec.6 slides

\section{MVC}
% pic in slides lec. 6

\section{Observer}

\begin{itemize}
  \item Observer and Observable java implementations
  \item Seperate model from the view
  \item slides lec.6
  \item java code lec. 6 slides
\end{itemize}


\end{document}